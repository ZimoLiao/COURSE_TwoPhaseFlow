%\documentclass{article}
\documentclass[11pt]{ctexart}

\usepackage{fancyhdr}
\usepackage{extramarks}
\usepackage{amsmath}
\usepackage{amsthm}
\usepackage{amsfonts}
\usepackage{tikz}
\usepackage[plain]{algorithm}
\usepackage{algpseudocode}
\usepackage{bm}
\usepackage{booktabs}
\usepackage{hyperref}
\usepackage{subfigure}

\usetikzlibrary{automata,positioning}

\def\equationautorefname{式}%
\def\footnoteautorefname{脚注}%
\def\itemautorefname{项}%
\def\figureautorefname{图}%
\def\tableautorefname{表}%
\def\partautorefname{篇}%
\def\appendixautorefname{附录}%
\def\chapterautorefname{章}%
\def\sectionautorefname{节}%
\def\subsectionautorefname{小小节}%
\def\subsubsectionautorefname{subsubsection}%
\def\paragraphautorefname{段落}%
\def\subparagraphautorefname{子段落}%
\def\FancyVerbLineautorefname{行}%
\def\theoremautorefname{定理}%

%
% Basic Document Settings
%

\topmargin=-0.45in
\evensidemargin=0in
\oddsidemargin=0in
\textwidth=6.5in
\textheight=9.0in
\headsep=0.25in

\linespread{1.1}

\pagestyle{fancy}
\lhead{\hmwkAuthorName}
\chead{\hmwkClass: \hmwkTitle}
\rhead{\firstxmark}
\lfoot{\lastxmark}
\cfoot{\thepage}

\renewcommand\headrulewidth{0.4pt}
\renewcommand\footrulewidth{0.4pt}

\setlength\parindent{0pt}

%
% Create Problem Sections
%

\newcommand{\enterProblemHeader}[1]{
    \nobreak\extramarks{}{问题 \arabic{#1} continued on next page\ldots}\nobreak{}
    \nobreak\extramarks{问题 \arabic{#1} (continued)}{问题 \arabic{#1} continued on next page\ldots}\nobreak{}
}

\newcommand{\exitProblemHeader}[1]{
    \nobreak\extramarks{问题 \arabic{#1} (continued)}{问题 \arabic{#1} continued on next page\ldots}\nobreak{}
    \stepcounter{#1}
    \nobreak\extramarks{问题 \arabic{#1}}{}\nobreak{}
}

\setcounter{secnumdepth}{0}
\newcounter{partCounter}
\newcounter{homeworkProblemCounter}
\setcounter{homeworkProblemCounter}{1}
\nobreak\extramarks{问题 \arabic{homeworkProblemCounter}}{}\nobreak{}

%
% Homework Problem Environment
%
% This environment takes an optional argument. When given, it will adjust the
% problem counter. This is useful for when the problems given for your
% assignment aren't sequential. See the last 3 problems of this template for an
% example.
%
\newenvironment{homeworkProblem}[1][-1]{
    \ifnum#1>0
        \setcounter{homeworkProblemCounter}{#1}
    \fi
    \section{问题 \arabic{homeworkProblemCounter}}
    \setcounter{partCounter}{1}
    \enterProblemHeader{homeworkProblemCounter}
}{
    \exitProblemHeader{homeworkProblemCounter}
}

\newenvironment{homeworkProblemSp}[1][-1]{
    \ifnum#1>0
        \setcounter{homeworkProblemCounter}{#1}
    \fi
    %\section{问题 \arabic{homeworkProblemCounter}}
    \setcounter{partCounter}{1}
    \enterProblemHeader{homeworkProblemCounter}
}{
    \exitProblemHeader{homeworkProblemCounter}
}

%
% Homework Details
%   - Title
%   - Due date
%   - Class
%   - Section/Time
%   - Instructor
%   - Author
%

\newcommand{\hmwkTitle}{作业\ 2}
\newcommand{\hmwkDueDate}{\today}
\newcommand{\hmwkClass}{格子玻尔兹曼方法}
%\newcommand{\hmwkClassTime}{}
%\newcommand{\hmwkClassInstructor}{}
\newcommand{\hmwkAuthorName}{廖紫默}
\newcommand{\hmwkAuthorID}{SA21005043}
\newcommand{\hmwkInstitution}{近代力学系,工程科学学院}

%
% Title Page
%

\title{
    \vspace{2in}
    \textmd{\textbf{\hmwkClass:\ \hmwkTitle}}\\
    \normalsize\vspace{0.1in}\hmwkDueDate
    %\vspace{0.1in}\large{\textit{\hmwkClassInstructor\ \hmwkClassTime}}
    \vspace{1in}
}

\author{\hmwkAuthorName\vspace{0.5ex}\\
\hmwkAuthorID\vspace{0.5ex}\\
\hmwkInstitution}
\date{}

\renewcommand{\part}[1]{\textbf{\large Part \Alph{partCounter}}\stepcounter{partCounter}\\}

%
% Various Helper Commands
%

% Useful for algorithms
\newcommand{\alg}[1]{\textsc{\bfseries \footnotesize #1}}

% For derivatives
\newcommand{\deriv}[1]{\frac{\mathrm{d}}{\mathrm{d}x} (#1)}

% For partial derivatives
\newcommand{\pderiv}[2]{\frac{\partial}{\partial #1} (#2)}

% Integral dx
\newcommand{\dx}{\mathrm{d}x}

% Alias for the Solution section header
\newcommand{\solution}{\textbf{\large Solution}}

% Probability commands: Expectation, Variance, Covariance, Bias
\newcommand{\E}{\mathrm{E}}
\newcommand{\Var}{\mathrm{Var}}
\newcommand{\Cov}{\mathrm{Cov}}
\newcommand{\Bias}{\mathrm{Bias}}

\begin{document}

\maketitle

\pagebreak

\begin{homeworkProblem}
    二维槽道流(Poiseuille流)模拟:

    \begin{enumerate}
        \item 施加体积力并采用modified bounce-back模拟壁面,实现Poiseuille流计算。计算多组参数并验证其空间精度是否为二阶;
        \item 将程序改编为最小存储形式,仅定义一套分布函数实现Streaming过程;
        \item 使用Zou-He(Non-equilibrium bounce-back)进出口边界条件实现压力驱动的Poiseuille流,同样要求验证其空间精度。
    \end{enumerate}

    \subsection{问题分解}

    \begin{itemize}
        \item 基本LBGK求解器
              \begin{itemize}
                  \item Streaming
                  \item Collision
                  \item Update moments and populations
                  \item I/O
              \end{itemize}
        \item 边界条件:
              \begin{itemize}
                  \item Half-way bounce-back
                  \item Modified bounce-back
                  \item Non-equilibrium (pressure/density) bounce-back [Zou-He]
              \end{itemize}
        \item 体积力
    \end{itemize}

    \subsection{1 边界条件的空间精度}

    计算了HBB(half-way bounce-back)和MBB(modified bounce-back)作为壁面边界条件时的Poiseuille流,检验了其2阶空间精度(HBB看作壁面正好位于两个lattice node中间的情况)。各算例参数选取如下\autoref{tab:bc}所示\footnote{在精度校验时需采用相同雷诺数,否则流动在物理上并非近似,使数值精度的判断有错误;另一方面还需注意$\tau$的选取,过小可能导致数值不稳定,但又需要限制$\tau$(在给定几何尺寸和雷诺数时)的大小以避免速度过大,偏离解析解的不可压假设。},其中$l_x,\ l_y$分别表示流向长度与槽道宽度。
    \begin{table}[htbp]
        \centering
        \caption{\label{tab:bc}边界条件精度测试参数}
        \begin{tabular}{p{3em}p{3em}p{3em}p{8em}}
            \toprule
            Re & $\tau$ & $l_x$  & $l_y$                  \\
            \midrule
            10 & 0.6    & $2l_y$ & $4,\ 8,\ 16,\ 32,\ 64$ \\
            \bottomrule
        \end{tabular}
    \end{table}

    采用槽道中线速度分布与解析解$\bm{u}^{(a)}$的均方误差与最大速度之比作为精度指标,即
    \begin{equation}
        \mathrm{err_r}\equiv \frac{1}{u_{\mathrm{max}}}\sqrt{\frac{1}{N}\sum_i \left(u_{i}-u_{i}^{(a)}\right)^2+\left(v_i-v_i^{(a)}\right)^2}
    \end{equation}
    \begin{table}[htbp]
        \centering
        \caption{\label{tab:bc_result}均方误差$\mathrm{err_r}$}
        \begin{tabular}{p{4em}p{6em}p{6em}p{6em}p{6em}p{6em}}
            \toprule
            B.C.   & $l_y=4$      & $l_y=8$      & $l_y=16$     & $l_y=32$     & $l_y=64$     \\
            \midrule
            HBB    & $6.7500e-02$ & $1.6875e-02$ & $4.2195e-03$ & $1.0561e-03$ & $2.6649e-04$ \\
            MBB    & $2.0000e-02$ & $4.9997e-03$ & $1.2493e-03$ & $3.1111e-04$ & $7.5288e-05$ \\
            NEBB-0 & $1.5875e-07$ & $3.5836e-07$ & $7.6136e-07$ & $1.5706e-06$ & $3.1910e-06$ \\
            \bottomrule
        \end{tabular}
    \end{table}

    如\autoref{fig:bc}所示,数值结果验证了HBB和MBB两种方法的二阶精度,从总体表现来看,NEBB-0精度最高\footnote{实际上NEBB-0,也就是在MBB基础上增加了$\pm\frac{1}{2}\left(f_2-f_4\right)$或$\pm\frac{1}{2}\left(f_1-f_3\right)$的修正,使得壁面(作为wet-node)速度严格等于0},MBB略优于HBB。
    \begin{figure}[htbp]
        \centering
        \includegraphics[width=0.45\linewidth]{figure/trial_bc.png}
        \caption{\label{fig:bc}\small Poiseuille流,三种边界条件的空间精度。其中NEBB-0数值计算结果在该构型下无限趋近于解析解,其精度仅受限于迭代步数(终止准则)与机器精度。}
    \end{figure}

    \subsection{2 最小存储形式}
    由于改编为最小存储形式需要对代码做较大规模重构,这里没有具体实现,大概阐述一下可行方法:\vspace{1em}

    用一套分布函数实现Streaming实际上是利用了Streaming过程的局域化特点,我们只需要在原始求解域边界外提供一圈buffer层,用于存储边界上会因Streaming过程而被覆盖的populations($f_i$),然后就可以边界向$\bar{\bm{c}}_i=-\bm{c}_i$方向逐步替换,即把$f_i(\bm{x},t)$赋值为$f_i(\bm{x}+\bar{\bm{c}}_i\Delta t,t)$,直至达到另一侧边界。显然这一操作对于不同方向$\bm{c}_i$是不同的,另外也需要针对分布函数存储的数据结构做具体的处理。

    \subsection{3 压力驱动Poiseuille流}
    采用Zou-He压力进出口(NEBB-P)边界,而非体积力,来模拟Poiseuille流,检验了NEBB的空间精度。这里上下无滑移壁面分别采用NEBB-0和HBB实现,测试结果证明两种情况下总体均有2阶空间精度(见\autoref{fig:zh}),均方误差见\autoref{tab:zh_result}。
    \begin{table}[htbp]
        \centering
        \caption{\label{tab:zh_result}均方误差$\mathrm{err_r}$}
        \resizebox{\textwidth}{!}{
            \begin{tabular}{p{10em}p{6em}p{6em}p{6em}p{6em}p{6em}}
                \toprule
                B.C.            & $l_y=4$      & $l_y=8$      & $l_y=16$     & $l_y=32$     & $l_y=64$     \\
                \midrule
                NEBB-P + HBB    & $6.3998e-02$ & $1.6525e-02$ & $4.2252e-03$ & $1.1044e-03$ & $3.2780e-04$ \\
                NEBB-P + NEBB-0 & $8.8062e-03$ & $2.4580e-03$ & $6.4197e-04$ & $1.6353e-04$ & $4.1311e-05$ \\
                \bottomrule
            \end{tabular}}
    \end{table}
    \begin{figure}[htbp]
        \centering
        \includegraphics[width=0.45\linewidth]{figure/trial_zh.png}
        \caption{\label{fig:zh}\small Zou-He压力进出口(即NEBB-P)空间精度,壁面采用HBB和NEBB-0,两者均具有2阶精度。}
    \end{figure}
    \begin{figure}[H]
        \centering
        \includegraphics[width=0.7\linewidth]{figure/nebb_nebb.png}
        \caption{\label{fig:channel}\small 压力驱动Poiseuille流(NEBB-P + NEBB-0),图中截取到宽度60位置,实际宽度为64。}
    \end{figure}

\end{homeworkProblem}

\pagebreak

\begin{homeworkProblem}
    槽道中圆柱绕流模拟,要求如下:

    \begin{enumerate}
        \item 给定速度入口(二维Poiseuille流解析解),出口采用简单外推边界,模拟槽道流中放置圆柱的流动问题;
        \item 计算圆柱的升力与阻力并做无量纲化得到对应的升/阻力系数,在非定常脱涡情况下计算升力系数振荡对应的Strouhal数。
    \end{enumerate}

    \subsection{问题分解}

    \begin{itemize}
        \item 边界条件:
              \begin{itemize}
                  \item Non-equilibrium (velocity) bounce-back [Zou-He]
                  \item Interpolated (Momentum transfer) bounce-back [Bouzidi]
                  \item Simple extrapolation
              \end{itemize}
        \item 升阻力计算及分析
              \begin{itemize}
                  \item 动量交换法
              \end{itemize}
    \end{itemize}

    \subsection{1 定常($\mathrm{Re}=20$)}
    入口采用NEBB-V给定速度剖面,出口采用1阶外推,上下无滑移壁面由NEBB-0实现,圆柱表面采用2阶IBB曲线边界。定常圆柱绕流的密度、涡量如下图所示:
    \begin{figure}[H]
        \centering
        \subfigure[$\rho$]{\includegraphics[width=\linewidth]{figure/steady_d40.png}}
        \subfigure[$\omega_z$]{\includegraphics[width=\linewidth]{figure/steady_d40_omega.png}}
        \caption{\label{fig:cylinder_s}\small 槽道中定常圆柱绕流($\mathrm{Re}=20,\ D=40$)密度、涡量云图}
    \end{figure}

    采用下述无量纲化:
    \begin{equation}
        C_L=\frac{F_L}{\frac{1}{2}\rho_0\bar{U}^2D},\quad C_D=\frac{F_D}{\frac{1}{2}\rho_0\bar{U}^2D}
    \end{equation}
    分别计算了圆柱直径$D=10,\ 20,\ 40$三种情况,与参考($D=80$)值对比如下表所示:
    \begin{table}[htbp]
        \centering
        \caption{\label{tab:cylinder_s}定常圆柱绕流升阻力系数,数值后为相对误差}
        \begin{tabular}{p{4em}p{4em}p{6em}p{10em}}
            \toprule
            case & $D$  & $C_D$              & $C_L$                           \\
            \midrule
            REF  & $80$ & $5.5650$           & $1.0460\times10^{-2}$           \\
            1    & $10$ & $6.7762\ (21.8\%)$ & $1.3968\times10^{-2}\ (33.5\%)$ \\
            2    & $20$ & $5.9410\ (6.7\%)$  & $1.1095\times10^{-2}\ (6.0\%)$  \\
            3    & $40$ & $5.7080\ (2.6\%)$  & $1.0220\times10^{-2}\ (2.3\%)$  \\
            \bottomrule
        \end{tabular}
    \end{table}

    可见随着直径增大,升阻力系数均逐渐趋近于参考解。

    \subsection{2 非定常($\mathrm{Re}=100$)}
    非定常圆柱绕流的密度、涡量如下图所示:
    \begin{figure}[H]
        \centering
        \subfigure[$\rho$]{\includegraphics[width=\linewidth]{figure/unsteady_d80_rho.png}}
        \subfigure[$\omega_z$]{\includegraphics[width=\linewidth]{figure/unsteady_d80_omega.png}}
        \caption{\label{fig:cylinder_u}\small 槽道中非定常圆柱绕流($\mathrm{Re}=20,\ D=80$)瞬时密度、涡量云图}
    \end{figure}

    圆柱所受升阻力见\autoref{fig:cylinder_force},最大升阻力系数、$\mathrm{St}$数如下表所示:
    \begin{table}[htbp]
        \centering
        \caption{\label{tab:cylinder_u}非定常圆柱绕流最大升阻力系数及(升力)$\mathrm{St}$数,数值后为相对误差}
        \begin{tabular}{p{4em}p{4em}p{6em}p{6em}p{6em}}
            \toprule
            case & $D$  & $C_{D,max}$       & $C_{L,max}$       & $\mathrm{St}$    \\
            \midrule
            REF  & $80$ & $3.291$           & $1.139$           & $0.292$          \\
            1    & $40$ & $3.930\ (19.4\%)$ & $1.248\ (9.6\%)$  & $0.302\ (3.4\%)$ \\
            2    & $60$ & $3.565\ (8.3\%)$  & $1.081\ (5.1\%)$  & $0.303\ (3.8\%)$ \\
            3    & $80$ & $3.428\ (4.2\%)$  & $0.976\ (14.3\%)$ & $0.304\ (4.1\%)$ \\
            \bottomrule
        \end{tabular}
    \end{table}

    \begin{figure}[H]
        \centering
        \subfigure[$D=40$]{\includegraphics[width=0.32\linewidth]{figure/D40U.png}}
        \subfigure[$D=60$]{\includegraphics[width=0.32\linewidth]{figure/D60U.png}}
        \subfigure[$D=80$]{\includegraphics[width=0.32\linewidth]{figure/D80U.png}}
        \caption{\label{fig:cylinder_force}\small 不同直径圆柱所受升阻力}
    \end{figure}

    结果与参考结果吻合较好,可能导致存在误差的原因有:
    \begin{enumerate}
        \item 这里我们采用了NEBB-0边界作为上下无滑移壁面,作业文档中参考解并未明确其使用的边界条件;
        \item 终止准则设置不当,这里我们给定了一定时间步,而非采用题给终止准则。
    \end{enumerate}


\end{homeworkProblem}

\pagebreak

\begin{homeworkProblem}

    \section{——理论概述与求解器设计}

    \subsection{控制方程与模型}

    控制方程采用以BGK模型作为碰撞算子的lattice Boltzmann方程,也称LBGK方程:
    \begin{equation}
        f_i(\bm{x}+\bm{c}_i\Delta t,t+\Delta t)=f_i(\bm{x},t)-\frac{1}{\tau}\left(f_i-f_i^{\mathrm{eq}}\right)\Delta t
    \end{equation}
    其中$\tau$为弛豫时间,平衡态分布函数由下式给出:
    \begin{equation}
        f_i^\mathrm{eq}(\bm{x},t)=w_i\rho\left(1+\frac{\bm{u}\cdot\bm{c}_i}{c_s^2}+\frac{(\bm{u}\cdot\bm{c}_i)^2}{2c_s^4}-\frac{\bm{u}\cdot\bm{u}}{2c_s^2}\right)
    \end{equation}
    其中$c_s$为声速。\vspace{1em}

    宏观变量,密度、速度等可以表示为分布函数的矩:
    \begin{eqnarray}
        \rho &=& \sum_i f_i=\sum_if_i^\mathrm{eq} \\
        \rho\bm{u} &=& \sum_i \bm{c}_if_i =\sum_i \bm{c}_if_i^\mathrm{eq}
    \end{eqnarray}
    而运动学粘性系数$\nu$由弛豫时间、声速等给出:
    \begin{equation}
        \nu=c_s^2\left(\tau-\frac{\Delta t}{2}\right)
    \end{equation}\vspace{1em}

    这里采用D2Q9模型,其速度离散格式如下\autoref{fig:d2q9}、\autoref{tab:d2q9}所示。
    \begin{figure}[htbp]
        \centering
        \includegraphics[width=0.2\linewidth]{figure/d2q9.png}
        \caption{\label{fig:d2q9}D2Q9速度集}
    \end{figure}
    \begin{table}[htbp]
        \centering
        \caption{\label{tab:d2q9}D2Q9速度集}
        \footnotesize
        \begin{tabular}{c|ccccccccc}
            \toprule
            $i$       & 0     & 1     & 2     & 3     & 4     & 5      & 6      & 7      & 8      \\
            \midrule
            $w_i$     & $4/9$ & $1/9$ & $1/9$ & $1/9$ & $1/9$ & $1/36$ & $1/36$ & $1/36$ & $1/36$ \\
            $c_{i,x}$ & $0$   & $+1$  & $0$   & $-1$  & $0$   & $+1$   & $-1$   & $-1$   & $+1$   \\
            $c_{i,y}$ & $0$   & $0$   & $+1$  & $0$   & $-1$  & $+1$   & $+1$   & $-1$   & $-1$   \\
            \bottomrule
        \end{tabular}
    \end{table}

    \subsection{无量纲化}

    采用格子单位对控制方程做无量纲化处理,若对任一有量纲量$v$,其无量纲形式记为$v^*=vC_v$,于是:
    \begin{equation}
        C_l=\Delta x,\ C_t=\Delta t,\ C_\rho=\rho\qquad\Longrightarrow\qquad\Delta x^*=1,\ \Delta t^*=1,\ \Delta \rho_0^*=1
    \end{equation}
    在格子单位下,无量纲声速$c_s^*=\sqrt{1/3}$,进而有无量纲运动学粘性系数$\nu^*=c_s^{*2}\left(\tau^*-1/2\right)$。雷诺数的表达式与有量纲时相同:
    \begin{equation}
        \mathrm{Re}=\frac{l^*U^*}{\nu^*}=\frac{lU}{\nu}
    \end{equation}

    \subsection{边界条件}
    求解器目前实现了多种边界条件,其类型标记如下所示:
    \begin{enumerate}
        \item \textit{F} fluid 流体单元 无特殊处理
        \item \textit{MBB} modified bounce-back
        \item \textit{GP} generalized periodic boundary [Kim \& Pitsch]
        \item \textit{SE} simple extrapolation outlet 含1、2阶
        \item \textit{NEBB-0} non-equilibrium bounce-back with zero velocity - no-slip wall [Zou-He]
        \item \textit{NEBB-V} non-equilibrium bounce-back with specified velocity [Zou-He]
        \item \textit{NEBB-P} non-equilibrium bounce-back with specified pressure/density [Zou-He]
        \item \textit{IBB} interpolated (momentum transfer) bounce-back [Bouzidi] 含1、2阶
        \item \textit{MS} mirror symmetry boundary 【后续计划】
    \end{enumerate}

    \subsubsection{一些要点及问题}

    \begin{enumerate}
        \item 角点处理:NEBB与HBB相交的角点,一方面得像NEBB一样streaming and colliding,一方面对于内部流体单元来说得按HBB一样处理(streaming向该点时反向回弹);
        \item 目前角点位置上的处理还是手动设置,后续有待改成自动识别并处理。
    \end{enumerate}

    \subsection{求解器架构设计}

    \subsubsection{求解器类}
    与上次大作业(拟压缩方法求解器)相似,定义了一个\textit{Lattice}类作为一块矩形域的求解器,由于LBM计算本身的局域化(Collision过程仅依赖于当地变量;Streaming过程仅依赖于当地格子及其相邻的8个格子单元)和显式推进等特性,未来可以很方便地扩展实现并行计算。\vspace{1em}

    \textit{Lattice}类中求解器核心分为以下四个部分:
    \begin{itemize}
        \item \textit{DoStreaming()} 实现Streaming过程,所有边界条件均在该函数内处理
        \item \textit{DoColliding()} 实现Collision过程,包括平衡态计算,完全局域化的计算流程
        \item \textit{DoForcing()} 施加体积力(源项)
        \item \textit{DoUpdate()} 更新分布函数(after streaming)与宏观变量(矩),同时计算终止准则条件
    \end{itemize}
    为了数值计算/程序编写过程的一致性,本求解器中所有涉及边界条件的操作均在Streaming过程实现,Collision过程对流体单元的操作完全一样。

    \subsubsection{流体单元抽象}
    LBM中最为关键的部分在于边界条件\footnote{不同边界条件用不同数值标记,这里借鉴了开源代码OpenLB的Material number思想}的实现与代码编写,为此,基于各类边界条件在具体实现时的根本差异,我们提出下面两种抽象:
    \begin{itemize}
        \item Fluid Boundary 流体边界
              \begin{itemize}
                  \item Wet wall 无滑移平直壁面(wet-node实现)
                  \item Inlet/Outlet 流体进出口边界,含:速度、压力边界,(含压力梯度的)周期边界,外推出口等
              \end{itemize}
        \item Non-Fluid Entity 实体
              \begin{itemize}
                  \item Wall/Obstacle 固定壁面(障碍物)
                  \item Element 可移动固体单元【后续计划,尚未实现单元运动的求解】
              \end{itemize}
    \end{itemize}
    两类流体单元抽象通过不同边界条件实现,前者所在lattice均为流体,参与所有的collision、streaming、forcing过程;后者所在lattice可以看作是在求解域之外,计算过程中并不需要对这些lattice做任何操作。基于这一抽象,定义了下述两个类:
    \begin{itemize}
        \item \textit{LatticeBound}
        \item \textit{LatticeEntity}
    \end{itemize}
    用于在\textit{Lattice}求解器种储存各流体单元的几何、边界类型、物理参数。于是我们可以通过在计算域种设置各种流体单元以实现各种构型的流场求解。此外,由于我们对各流体单元做了标记,在处理边界条件时,可以直接利用流体单元索引需特殊处理的部分,而不用全域搜索判断边界条件类型,从而提高计算效率。


    \subsection{扩展算例}
    基于已有求解器,可以很方便实现各种流动构型的计算,简单的如后台阶、方腔流、多几何体绕流等。

    \begin{figure}[htbp]
        \centering
        \includegraphics[width=0.9\linewidth]{figure/extension_stepflow.png}
        \caption{后台阶流动}
    \end{figure}

    \begin{figure}[htbp]
        \centering
        \includegraphics[width=0.5\linewidth]{figure/extension_cavity.png}
        \caption{顶盖驱动方腔流($Re=1000$),边界条件采用NEBB}
    \end{figure}

    \begin{figure}[htbp]
        \centering
        \includegraphics[width=0.9\linewidth]{figure/extension_tricylinder.png}
        \caption{三个交错排列的圆柱绕流}
    \end{figure}

\end{homeworkProblem}


\end{document}