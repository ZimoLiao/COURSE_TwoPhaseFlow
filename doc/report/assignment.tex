\documentclass[10pt]{article}
\usepackage{xeCJK}
\usepackage{geometry}
\usepackage{bm}
\usepackage{amsmath}
\usepackage{graphicx}
\usepackage{hyperref}
\usepackage{subfigure}
\usepackage{datetime}
\usepackage{fontspec}
\usepackage{fancyhdr}
\usepackage{indentfirst}
\usepackage{titlesec}

% global style
\geometry{b5paper,left=0.6in,right=0.6in,top=1in,bottom=0.8in}
\linespread{1.3}
\setlength{\parindent}{2em}

\renewcommand{\today}{\small \number\year 年 \number\month 月 \number\day 日}
\renewcommand{\abstractname}{\normalsize 摘要}
\renewcommand{\contentsname}{目录}
\renewcommand{\refname}{参考文献}
\renewcommand{\sectionmark}[1]{\markright{第\,\thesection\,章\, #1}}

% title style
\titleformat*{\section}{\Large\bfseries}
\titleformat*{\subsection}{\large\bfseries}
\titleformat*{\subsubsection}{\normalsize\bfseries}

% font style
\setCJKmainfont{SimSun}[BoldFont=SimHei, ItalicFont=KaiTi]
\setmainfont{Times New Roman}[BoldFont=Arial]

% page style
\fancypagestyle{assignment-title}{
    \fancyhf{}
    \fancyhead[L]{\small 廖紫默 (SA21005043)}
    \fancyhead[C]{\small \textbf{多相流大作业}}
    \fancyhead[R]{\small 2021年秋}
    \renewcommand{\headrulewidth}{0.4pt}
    \renewcommand{\footrulewidth}{0.4pt}
}
\fancypagestyle{assignment}{
    \fancyhf{}
    \fancyfoot[C]{\thepage}
    \fancyhead[L]{\small 廖紫默 (SA21005043)}
    \fancyhead[C]{\small \textbf{多相流大作业}}
    \fancyhead[R]{\small \bfseries\rightmark}
    \renewcommand{\headrulewidth}{0.4pt}
    \renewcommand{\footrulewidth}{0.4pt}
}
\pagestyle{assignment}

% ref style (in chinese)
\def\equationautorefname{式}%
\def\footnoteautorefname{脚注}%
\def\itemautorefname{项}%
\def\figureautorefname{图}%
\def\tableautorefname{表}%
\def\partautorefname{篇}%
\def\appendixautorefname{附录}%
\def\chapterautorefname{章}%
\def\sectionautorefname{节}%
\def\subsectionautorefname{小小节}%
\def\subsubsectionautorefname{subsubsection}%
\def\paragraphautorefname{段落}%
\def\subparagraphautorefname{子段落}%
\def\FancyVerbLineautorefname{行}%
\def\theoremautorefname{定理}%


% header
\title{\LARGE\textbf{多相流大作业报告}}
\author{廖紫默 (SA21005043)\\
\small 近代力学系,中国科学技术大学\\
\small \href{mailto:zimoliao@mail.ustc.edu.cn}{zimoliao@mail.ustc.edu.cn}}
\date{\today}

\begin{document}

% titlepage
\maketitle
\thispagestyle{assignment-title}
\pagenumbering{roman}
\begin{abstract}
    \normalsize\it 一无是处的摘要一无是处的摘要一无是处的摘要一无是处的摘要一无是处的摘要一无是处的摘要一无是处的摘要一无是处的摘要一无是处的摘要一无是处的摘要一无是处的摘要一无是处的摘要一无是处的摘要一无是处的摘要一无是处的摘要
\end{abstract}

\tableofcontents
% titlepage end


%part 1
\newpage
\pagenumbering{arabic}
\setcounter{page}{1}
\section{引言}


% part 2
\newpage
\section{Level Set方法}
\subsection{基本思想:隐式界面}
\textbf{ancyhdr} \textit{宏包niubi}\textbf{改善了页眉页脚样式为Arial}的定义方式,允许我们将内容自由安置在页眉和页脚的
左、中、右三个位置,还为页眉和页脚各加了一条横线。
fancyhdr 自定义了样式名称 fancy。使用 fancyhdr 宏包定义页眉页脚之前,通常先用  调用这个样式。在 fancyhdr 中定义页眉页脚的命令为

\subsection{界面描述:符号距离函数}
首先我们定义(欧式)\textbf{距离函数(distance function)}:
\begin{equation}
    d(\bm{x})=\min\left(\left|\bm{x}-\bm{x}_I\right|\right)\quad\text{for all}\quad \bm{x}_I\in\partial\Omega
\end{equation}
显然$d(\bm{x})\geq0$,且其梯度的模在除某些奇异点外均为$1$,具有很好的光滑性:
\begin{eqnarray}
    \left|\nabla d\right| &=& \sqrt{\left[\frac{\partial}{\partial x_i}\sqrt{\left(x_l-x_{I,l}\right)\left(x_l-x_{I,l}\right)}\right]\left[\frac{\partial}{\partial x_i}\sqrt{\left(x_l-x_{I,l}\right)\left(x_l-x_{I,l}\right)}\right]}\notag\\
    &=&\sqrt{\frac{1}{2}\frac{2\left(x_i-x_{I,i}\right)}{\sqrt{\left(x_l-x_{I,l}\right)\left(x_l-x_{I,l}\right)}}\cdot\frac{1}{2}\frac{2\left(x_i-x_{I,i}\right)}{\sqrt{\left(x_l-x_{I,l}\right)\left(x_l-x_{I,l}\right)}}}\notag\\
    &=&\sqrt{\frac{\left(x_i-x_{I,i}\right)\left(x_i-x_{I,i}\right)}{\left(x_l-x_{I,l}\right)\left(x_l-x_{I,l}\right)}}\notag\\
    &=&1
\end{eqnarray}
上式采用了爱因斯坦求和约定。

在此基础上定义\textbf{符号距离函数(signed distance function)}$\phi(\bm{x})$,使得$|\phi(\bm{x})|=d(\bm{x})$,并要求它满足隐式界面的定义,具体表达式如下:
\begin{eqnarray}
    \phi(\bm{x})=\left\{\begin{array}{rl}
        d(\bm{x})   & \text{for all}\ \bm{x}\in\Omega^+       \\
        d(\bm{x})=0 & \text{for all}\ \bm{x}\in\partial\Omega \\
        -d(\bm{x})  & \text{for all}\ \bm{x}\in\Omega^-       \\
    \end{array}\right.
    \label{eqn:sdf}
\end{eqnarray}
选取符号距离函数作为隐式界面函数具有十分优良的性质,例如其在界面附近的光滑性,及常梯度幅值$\left|\phi(\bm{x})\right|=1$对于界面曲率计算的大大简化:
\begin{equation}
    \kappa\equiv\nabla\cdot\bm{n}=\nabla^2\phi,\quad \bm{n}=\nabla\phi
\end{equation}

\subsection{界面推进}
自由界面问题中,通常假设两相界面为一物质面,隐式界面函数(这里也就是符号距离函数)$\phi(\bm{x})$的物质导数为零即可得到其演化方程:
\begin{equation}
    \frac{\mathrm{D}\phi}{\mathrm{D}t}=\frac{\partial\phi}{\partial t}+\bm{u}\cdot\nabla\phi=0\quad\text{for all }\bm{x}\in\partial\Omega
    \label{eqn:evolution}
\end{equation}
其中$\bm{u}(x,t)$为界面处流场速度,注意该式理论上仅在界面位置(即$\bm{x}_I:\phi(\bm{x}_I)=0$)成立。

\subsubsection{时间推进格式}
数值求解\autoref{eqn:evolution}时,时间推进的最简单实现莫过于一阶精度的显式格式——\textbf{前向欧拉方法(forward Euler method)},如下式:
\begin{equation}
    \frac{\phi^{n+1}-\phi^n}{\Delta t}+\bm{u}^n\cdot\nabla\phi^n=0
\end{equation}
其中上标表所处时间步。

考虑进一步提高时间精度,可以采用\textbf{TVD RK(total variation diminishing Runge-Kutta)格式},其二阶精度版本等同于常规RK2格式:
\begin{eqnarray}
    \frac{\phi^{n+1}-\phi^n}{\Delta t}+\bm{u}^n\cdot\nabla\phi^n=0 \\
    \frac{\phi^{n+2}-\phi^{n+1}}{\Delta t}+\bm{u}^{n+1}\cdot\nabla\phi^{n+1}=0 \\
    \phi^{n+1}=\frac{1}{2}\left(\phi^n+\phi^{n+2}\right)
\end{eqnarray}

三阶精度的TVD RK格式如下:
\begin{eqnarray}
    \frac{\phi^{n+1}-\phi^n}{\Delta t}+\bm{u}^n\cdot\nabla\phi^n=0 \\
    \frac{\phi^{n+2}-\phi^{n+1}}{\Delta t}+\bm{u}^{n+1}\cdot\nabla\phi^{n+1}=0 \\
    \phi^{n+\frac{1}{2}}=\frac{3}{4}\phi^n+\frac{1}{4}\phi^{n+2} \\
    \frac{\phi^{n+\frac{3}{2}}-\phi^{n+\frac{1}{2}}}{\Delta t}+\bm{u}^{n+\frac{1}{2}}\cdot\nabla\phi^{n+\frac{1}{2}}=0 \\
    \phi^{n+1}=\frac{1}{3}\phi^{n+\frac{1}{2}}+\frac{2}{3}\phi^{n+\frac{3}{2}}
\end{eqnarray}
更高阶的时间推进格式对于提升界面捕捉的精度收效甚微,上述三种格式对于常规问题已经足够。

\subsubsection{空间离散格式}

% part 3
\newpage
\section{Volume of Fluid方法}

% part 4
\newpage
\section{Phase Field方法}

% part 5
\newpage
\section{算例验证与分析}
\subsection{缺角圆盘旋转}

\subsection{液滴在剪切流中变形}

\bibliographystyle{plain}
\bibliography{ref}

\end{document}